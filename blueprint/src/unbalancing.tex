\chapter{Unbalancing}

\begin{lemma}
\label{nonneg_moments}
\lean{pow_inner_nonneg}
For any function $f:G\to\bbr$ and integer $k\geq 0$
\[\bbe_x f\circ f(x)^k\geq 0.\]
\end{lemma}
\begin{proof}
\leanok
Test.
\end{proof}

\begin{lemma}
\label{unbalancing}
\lean{unbalancing}
\leanok
Let $\epsilon\in(0,1)$ and $\nu:G\to\bbr_{\geq 0}$ be some probability measure such that $\widehat{\nu}\geq 0$. Let $f:G\to\bbr$. If $\norm{f\circ f}_{p(\nu)}\geq \epsilon$ for some $p\geq 1$ then $\norm{f\circ f+1}_{p'(\nu)}\geq 1+\tfrac{1}{2}\epsilon$ for some $p'\ll_\epsilon p$.
\end{lemma}
\begin{proof}
\uses{nonneg_moments}
Without loss of generality we can assume that $p\geq 5$ is an odd integer. % Want (3/4)^p \leq 1/4
Since the Fourier transforms of $f$ and $\nu$ are non-negative,
\[\bbe \nu f^{p}= \widehat{\nu}\ast\widehat{f}^{(p)}(0_{\widehat{G}})\geq 0.\]
It follows that, since $2\max(x,0)=x+\abs{x}$ for $x\in\bbr$,
\[2\langle \max(f,0),f^{p-1}\rangle_\nu=\bbe \nu f^{p}+\langle \abs{f},f^{p-1}\rangle_\nu\geq \norm{f}_{p(\nu)}^p\geq \epsilon^p.\]
Therefore, if $P=\{ x : f(x) \geq 0\}$, then $\langle \ind{P},f^{p}\rangle_\nu\geq \frac{1}{2}\epsilon^{p}$. Furthermore, if $T=\{ x\in P : f(x) \geq \tfrac{3}{4}\epsilon\}$ then $\langle \ind{P\backslash T},f^p\rangle_\nu \leq \tfrac{1}{4}\epsilon^{p}$, and hence by the Cauchy-Schwarz inequality,
\[\nu(T)^{1/2}\norm{f}_{2p(\nu)}^{p}\geq \langle \ind{T}, f^{p}\rangle_\nu \geq \tfrac{1}{4}\epsilon^{p}.\]
On the other hand, by the triangle inequality
\[\norm{f}_{2p(\nu)}\leq 1+\norm{f+1}_{2p(\nu)}\leq 3,\]
or else we are done, with $p'=2p$. Hence $\nu(T)\geq (\epsilon/10)^{2p}$. It follows that, for any $p'\geq 1$,
\[\norm{f+1}_{p'(\nu)}\geq \langle \ind{T},\abs{f+1}^{p'}\rangle_\nu^{1/p'}\geq  (1+\tfrac{3}{4}\epsilon)(\epsilon/10)^{2p/p'}.\]
The desired bound now follows if we choose $p'$ a sufficiently large multiple (depending on $\epsilon$) of $p$.
\end{proof}
