\chapter{Chang's lemma}

\begin{definition}[Dissociation]
\label{dissociated}
\lean{AddDissociated}
\leanok
We say that $A\subseteq G$ is dissociated if, for any $m\geq 1$, and any $x\in G$, there is at most one $A'\subset A$ of size $\abs{A'}=m$ such that
\[\sum_{a\in A'}a=x.\]
\end{definition}


\begin{lemma}[Rudin's exponential inequality]
\label{rudin_exp}
\uses{dissociated}
\lean{rudin_exp_ineq}
\leanok
If the discrete Fourier transform of $f : G \longrightarrow \C$ has dissociated support, then
$$\E \exp(\Re f) \le \exp\left(\frac{\norm f_2^2} 2\right)$$
It follows that
\[\E_x e^{\abs{f(x)}} \le 2e^{\| f\|_2^2/2}.\]
\end{lemma}
\begin{proof}
\uses{mzi_complex}
\leanok
Using the convexity of $t\mapsto e^{tx}$ (for all $x\geq 0$ and $t\in[-1,1]$) we have
\[e^{tx}\le \cosh(x)+t\sinh(x).\]
It follows (taking $x=\lvert z\rvert$ and $t=\Re(z)/\lvert z\rvert$) that, for any $z\in \mathbb{C}$,
\[e^{\Re z}\le \cosh(\lvert z\rvert)+\Re(z/\lvert z\rvert)\sinh(\lvert z\rvert).\]
In particular, if $c_\gamma\in \mathbb{C}$ with $\lvert c_\gamma\rvert=1$ is such that $\widehat{f}(\gamma)=c_\gamma\lvert \widehat{f}(\gamma)\rvert$, then
\begin{align*}
e^{\Re f(x)}
&= \exp\left( \Re \sum_{\gamma\in\Gamma}\widehat{f}(\gamma)\gamma(x)\right)\\
&=\prod_{\gamma\in \Gamma} \exp\left( \Re \widehat{f}(\gamma)\gamma(x)\right)\\
&\le \prod_{\gamma\in \Gamma}\left( \cosh(\lvert \widehat{f}(\gamma)\rvert)+\Re c_\gamma \gamma(x)\sinh(\lvert \widehat{f}(\gamma)\rvert)\right).
\end{align*}
Therefore
\[\E_x e^{\Re f(x)}\le \E_x \prod_{\gamma\in \Gamma} \left( \cosh(\lvert \widehat{f}(\gamma)\rvert)+\Re c_\gamma \gamma(x)\sinh(\lvert \widehat{f}(\gamma)\rvert)\right).\]
Using $\Re z=(z+\overline{z})/2$ the product here can be expanded as the sum of
\[\prod_{\gamma\in \Gamma_2}\frac{c_\gamma}{2}\prod_{\gamma\in \Gamma_3}\frac{\overline{c_\gamma}}{2}\left(\prod_{\gamma\in \Gamma_1}\cosh(\lvert \widehat{f}(\gamma)\rvert)\right)\left(\prod_{\gamma\in \Gamma_2\cup\Gamma_3}\sinh(\lvert \widehat{f}(\gamma)\rvert)\right)\left(\sum_{\gamma\in \Gamma_2}\gamma-\sum_{\lambda\in \Gamma_3}\lambda\right)(x)\]
as $\Gamma_1\sqcup \Gamma_2\sqcup \Gamma_3=\Gamma$ ranges over all partitions of $\Gamma$ into three disjoint parts. Using the definition of dissociativity we see that
\[\sum_{\gamma\in \Gamma_2}\gamma-\sum_{\lambda\in \Gamma_3}\lambda\neq 0\]
unless $\Gamma_2=\Gamma_3=\emptyset$. In particular summing this term over all $x\in G$ gives $0$. Therefore the only term that survives averaging over $x$ is when $\Gamma_1=\Gamma$, and so
\[\E_x e^{\Re f(x)}\le \prod_{\gamma\in \Gamma} \cosh (\lvert \widehat{f}(\gamma)\rvert).\]
The conclusion now follows using $\cosh(x) \le e^{x^2/2}$ and $\sum_{\gamma\in \Gamma}\lvert \widehat{f}(\gamma)\rvert^2=\| f\|_2^2$. The second conclusion follows by applying it to $f(x)$ and $-f(x)$ and using
\[e^{\abs{y}}\le e^y+e^{-y}.\]
\end{proof}


\begin{lemma}[Rudin's inequality]
\label{rudin}
\uses{dissociated}
\lean{rudin_ineq}
\leanok
If the discrete Fourier transform of $f : G \longrightarrow \C$ has dissociated support and $p \ge 2$ is an integer, then $\norm{f}_p \le 4\sqrt{pe} \norm f_2$.
\end{lemma}
\begin{proof}
\uses{rudin_exp}
\leanok
It is enough to show that $\norm{\Re f}_p \le 2\sqrt{pe} \norm f_2$ as then
$$\norm{f}_p \le \norm{\Re f}_p + \norm{i \Im f}_p = \norm{\Re f}_p + \norm{\Re (-if)}_p \le 4\sqrt{pe} \norm f_2$$

If $f = 0$, the result is obvious. So assume $f \ne 0$. $\norm{f}_2 > 0$, so WLOG $\norm{f}_2 = \sqrt p$.

Rudin's exponential inequality for $f$ becomes $\E \exp|\Re f| \le 2\exp(\frac p 2) = (2\sqrt e)^p$. Using $\frac{x^p}{p!} \le e^x$ for positive $x$, we get
$$\frac{\norm{\Re f}_p^p}{p^p} \le \frac{\norm{\Re f}_p^p}{p!} = \E \frac{|\Re f|^p}{p!} \le \E \exp|\Re f|$$
Rearranging, $\norm{\Re f}_p \le 2 p\sqrt{e} = 2 \sqrt{pe} \norm f_2$.
% TODO: Normalisation issue
\end{proof}


\begin{definition}[Large spectrum]
\label{large_spec}
\lean{large_spec}
\leanok
Let $G$ be a finite abelian group and $f:G\to\bbc$. Let $\eta\in \bbr$. The $\eta$-large spectrum is defined to be
\[\Delta_\eta(f) = \{ \gamma\in\widehat{G} : \lvert \widehat{f}(\gamma)\rvert \geq \eta\norm{f}_1\}.\]
\end{definition}


\begin{definition}[Weighted energy]
\label{weight_energy}
\lean{energy}
\leanok
Let $\Delta\subseteq \widehat{G}$ and $m\geq 1$. Let $\nu:G\to \bbc$. Then
\[E_{2m}(\Delta;\nu)=\sum_{\gamma_1,\ldots,\gamma_{2m}\in \Delta}\abs{\widehat{\nu}(\gamma_1+\cdots-\gamma_{2m})}.\]
\end{definition}


\begin{definition}[Energy]
\label{energy}
\uses{weight_energy}
\lean{boringEnergy}
\leanok
Let $G$ be a finite abelian group and $A\subseteq G$. Let $m\geq 1$. We define
\[E_{2m}(A)=\sum_{a_1,\ldots,a_{2m}\in A}1_{a_1+\cdots-a_{2m}=0}.\]
\end{definition}


\begin{lemma}
\label{general_hoelder}
\uses{large_spec, weight_energy}
\lean{general_hoelder}
\leanok
Let $G$ be a finite abelian group and $f:G\to\bbc$. Let $\nu:G\to \bbr_{\geq 0}$ be such that whenever $\abs{f}\neq 0$ we have $\nu \geq 1$. Let $\Delta\subseteq \Delta_\eta(f)$. Then, for any $m\geq 1$.
\[\eta^{2m}\frac{\norm{f}_1^2}{\norm{f}_2^2}\abs{\Delta}^{2m}\le E_{2m}(\Delta;\nu).\]
\end{lemma}

\begin{proof}
\leanok
By definition of $\Delta_\eta(f)$ we know that
\[\eta\norm{f}_1\abs{\Delta}\le \sum_{\gamma\in\Delta} \lvert \widehat{f}(\gamma)\rvert.\]
There exists some $c_\gamma\in\bbc$ with $\lvert c_\gamma\rvert=1$ for all $\gamma$ such that
\[\lvert \widehat{f}(\gamma)\rvert=c_\gamma\widehat{f}(\gamma)=c_\gamma \sum_{x\in G}f(x)\overline{\gamma(x)}.\]
Interchanging the sums, therefore,

\[\eta\norm{f}_1\abs{\Delta}\le \sum_{x\in G}f(x)\sum_{\gamma\in\Delta} c_\gamma \overline{\gamma(x)}.\]
By H\"{o}lder's inequality the right-hand side is at most
\[\brac{\sum_x \abs{f(x)}}^{1-1/m}\brac{\sum_x \abs{f(x)}\abs{\sum_{\gamma\in\Delta}c_\gamma\overline{\gamma(x)}}^m}^{1/m}.\]
Taking $m$th powers, therefore, we have
\[\eta^m\abs{\Delta}^m\norm{f}_1\le \sum_{x}\abs{f(x)}\abs{\sum_{\gamma\in\Delta}c_\gamma\overline{\gamma(x)}}^m.\]
By assumption we can bound $\abs{f(x)}\le \abs{f(x)}\nu(x)^{1/2}$, and therefore by the Cauchy-Schwarz inequality the right-hand side is bounded above by
\[\norm{f}_2\brac{\sum_x \nu(x)\abs{\sum_{\gamma\in\Delta}c_\gamma\overline{\gamma(x)}}^{2m}}^{1/2}.\]
Squaring and simplifying, we deduce that
\[\eta^{2m}\abs{\Delta}^{2m}\frac{\norm{f}_1^2}{\norm{f}_2^2}\le \sum_x \nu(x)\abs{\sum_{\gamma\in\Delta}c_\gamma\overline{\gamma(x)}}^{2m}.\]
Expanding out the power, the right-hand side is equal to
\[\sum_x \nu(x)\sum_{\gamma_1,\ldots,\gamma_{2m}}c_{\gamma_1}\cdots \overline{c_{\gamma_{2m}}} (\overline{\gamma_1}\cdots \gamma_{2m})(x).\]
Changing the order of summation this is equal to
\[\sum_{\gamma_1,\ldots,\gamma_{2m}}c_{\gamma_1}\cdots \overline{c_{\gamma_{2m}}}
\widehat{\nu}(\gamma_1\cdots \overline{\gamma_{2m}}).\]
The result follows by the triangle inequality.
\end{proof}


\begin{lemma}
\label{spec_hoelder}
\uses{energy}
\lean{spec_hoelder}
\leanok
Let $G$ be a finite abelian group and $f:G\to\bbc$. Let $\Delta\subseteq \Delta_\eta(f)$. Then, for any $m\geq 1$.
\[N^{-1}\eta^{2m}\frac{\norm{f}_1^2}{\norm{f}_2^2}\abs{\Delta}^{2m}\le E_{2m}(\Delta).\]
\end{lemma}

\begin{proof}
\uses{general_hoelder}
\leanok
Apply Lemma~\ref{general_hoelder} with $\nu\equiv 1$, and use the fact that $\sum_x \lambda(x)$ is $N$ if $\lambda\equiv 1$ and $0$ otherwise.
\end{proof}


\begin{lemma}
\label{energy_alt}
\uses{energy}
\lean{boringEnergy_eq}
\leanok
If $A\subset G$ and $m\geq 1$ then
\[E_{2m}(A) = \sum_x 1_A^{(m)}(x)^2.\]
\end{lemma}

\begin{proof}
\leanok
Expand out definitions.
\end{proof}


\begin{lemma}
\label{diss_energy}
\uses{dissociated, energy}
\lean{AddDissociated.boringEnergy_le}
\leanok
If $A\subseteq G$ is dissociated then $E_{2m}(A) \le (32e m \abs{A})^m$.
\end{lemma}
\begin{proof}
\uses{energy_alt, rudin}
\leanok
By Lemma~\ref{energy_alt} and Lemma~\ref{rudin}
\begin{align*}
  E_{2m}(A)
  & = \E_\gamma \abs{\hat 1_A(\gamma)}^{2m} \\
  & = \norm{\hat 1_A}_{2m}^{2m} \\
  & \le (4\sqrt{2em})^{2m} \norm{\hat 1_A}_2^{2m} \\
  & = (32em)^m \norm{1_A}_2^{2m} \\
  & = (32em)^m \abs{A}^m
\end{align*}
\end{proof}


\begin{lemma}
\label{diss_span}
\uses{dissociated}
\lean{finset.diss_span}
\leanok
If $A\subseteq G$ contains no dissociated set with $\geq K+1$ elements then there is $A'\subseteq A$ of size $\abs{A'}\le K$ such that
\[A\subseteq \left\{ \sum_{a\in A'}c_aa : c_a\in \{-1,0,1\} \right\}.\]
\end{lemma}

\begin{proof}
\leanok
Let $A'\subseteq A$ be a maximal dissociated subset (this exists and is non-empty, since trivially any singleton is dissociated). We have $\abs{A'}\le K$ by assumption.

Let $S$ be the span on the right-hand side. It is obvious that $A'\subseteq S$. Suppose that $x\in A\backslash A'$. Then $A'\cup\{x\}$ is not dissociated by maximality. Therefore there exists some $y\in G$ and two distinct sets $B,C\subseteq A'\cup \{x\}$ such that
\[\sum_{b\in B}b = y = \sum_{c\in C} c.\]
If $x\not\in B$ and $x\not\in C$ then this contradicts the dissociativity of $A'$. If $x\in B$ and $x\in C$ then we have
\[\sum_{b\in B\backslash x}b=y-x=\sum_{c\in C\backslash x}c,\]
again contradicting the dissociativity of $A'$. Without loss of generality, therefore, $x\in B$ and $x\not\in C$. Therefore
\[x=\sum_{c\in C}c - \sum_{b\in B\backslash x}b\]
which is in the span as required.
\end{proof}


\begin{theorem}[Chang's lemma]
\label{chang}
\uses{large_spec}
\lean{chang}
\leanok
Let $G$ be a finite abelian group and $f:G\to \bbc$. Let $\eta >0$ and $\alpha=N^{-1}\norm{f}_1^2/\norm{f}_2^2$. There exists some $\Delta\subseteq \Delta_\eta(f)$ such that
\[\abs{\Delta}\le \lceil e\mathcal{L}(\alpha)\eta^{-2}\rceil \]
and
\[\Delta_\eta(f)\subseteq \left\{ \sum_{\gamma\in\Delta}c_\gamma \gamma : c_\gamma\in \{-1,0,1\} \right\}.\]
\end{theorem}

\begin{proof}
\uses{diss_energy, diss_span, spec_hoelder}
\leanok
By Lemma~\ref{diss_span} it suffices to show that $\Delta_\eta(f)$ contains no dissociated set with at least
\[K= \lceil e\mathcal{L}(\alpha)\eta^{-2}\rceil+1\]
many elements. Suppose not, and let $\Delta\subseteq \Delta_\eta(f)$ be a dissociated set of size $K$. Then by Lemma~\ref{diss_energy} we have, for any $m\geq 1$,
\[E_{2m}(\Delta)\le m!K^m.\]
On the other hand, by Lemma~\ref{spec_hoelder},

\[\eta^{2m}\alpha K^{2m}\le E_{2m}(\Delta).\]
Rearranging these bounds, we have
\[K^m \le m! \alpha^{-1}\eta^{-2m}\le m^m\alpha^{-1}\eta^{-2m}.\]
Therefore $K\le \alpha^{-1/m}m\eta^{-2}$. This is a contradiction to the choice of $K$ if we choose $m=\mathcal{L}(\alpha)$, since $\alpha^{-1/m}\le e$.
\end{proof}
