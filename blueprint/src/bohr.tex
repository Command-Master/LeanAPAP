\chapter{Bohr sets}

\begin{definition}[Bohr sets]
For a non-empty $\Gamma\subseteq \widehat{G}$ and $\nu\in [0,2]$ we define the Bohr set $B=\mathrm{Bohr}_\nu(\Gamma)$ as 
\[\mathrm{Bohr}_\nu(\Gamma)=\left\{ x\in G : \abs{1-\gamma(x)}\leq \nu\textrm{ for all }\gamma\in\Gamma\right\}.\]
We call $\Gamma$ the \emph{frequency set} of $B$ and $\nu$ the \emph{width}, and define the \emph{rank} of $B$ to be the size of $\Gamma$, denoted by $\rk(B)$. We note here that all Bohr sets are symmetric and contain $0$.

In fact, when we speak of a Bohr set we implicitly refer to the triple $(\Gamma,\nu,\mathrm{Bohr}_\nu(\Gamma))$, since the set $\mathrm{Bohr}_{\nu}(\Gamma)$ alone does not uniquely determine the frequency set nor the width. When we use subset notation, such as $B'\subseteq B$, this refers only to the set inclusion (and does not, in particular, imply any particular relation between the associated frequency sets or width functions). Furthermore, if $B=\mathrm{Bohr}_\nu(\Gamma)$ and $\rho\in(0,1]$, then we write $B_\rho$ for the same Bohr set with the width dilated by $\rho$, i.e. $\mathrm{Bohr}_{\rho\nu}(\Gamma)$, which is known as a \emph{dilate} of $B$.
\end{definition}

\begin{definition}[Regularity]
A Bohr set $B$ of rank $d$ is regular if for all $\abs{\kappa}\leq 1/100d$ we have 
\[(1-100 d\abs{\kappa})\abs{B}\leq \abs{B_{1+\kappa}}\leq(1+ 100 d\abs{\kappa})\abs{B}.\]
\end{definition}

\begin{lemma}\label{bohr-regular}
For any Bohr set $B$ there exists $\rho\in[\tfrac{1}{2},1]$ such that $B_\rho$ is regular.
\end{lemma}
\begin{proof}
To do.
\end{proof}

\begin{lemma}\label{bohr-size}
If $\rho\in (0,1)$ and $B$ is a Bohr set of rank $d$, then $\abs{B_\rho}\geq (\rho/4)^d\abs{B}$.
\end{lemma}
\begin{proof}
To do.
\end{proof}

\begin{lemma}\label{reg-conv}
If $B$ is a regular Bohr set of rank $d$ and $\mu:G\to\bbr_{\geq 0}$ is supported on $B_\rho$, with $\rho \in (0,1)$, then
\[ \norm{ \mu_B*\mu - \mu_B }_{1} \ll \rho d\norm{\mu}_1. \]
\end{lemma}
\begin{proof}
To do.
\end{proof}

\begin{lemma}\label{bohr-majorise}
There is a constant $c>0$ such that the following holds.  Let $B$ be a regular Bohr set of rank $d$ and $L\geq 1$ be any integer. If $\nu:G\to\bbr_{\geq 0}$ is supported on $L B_\rho$, where $\rho \leq c/Ld$, and $\norm{\nu}_1=1$, then 
\[\mu_B \leq 2\mu_{B_{1+L\rho}}\ast \nu.\]
\end{lemma}
\begin{proof}
\uses{reg-conv}
To do.
\end{proof}

\begin{lemma}\label{bourgain-trick}
There is a constant $c>0$ such that the following holds. Let $B$ be a regular Bohr set of rank $d$, suppose $A\subseteq B$ has density $\alpha$, let $\epsilon>0$, and suppose $B',B''\subseteq B_\rho$ where $\rho\leq c\alpha\epsilon/d$. Then either
\begin{enumerate}
\item there is some translate $A'$ of $A$ such that $\abs{A'\cap B'}\geq (1-\epsilon)\alpha\abs{B'}$ and $\abs{A'\cap B''}\geq (1-\epsilon)\alpha \abs{B''}$, or 
\item $\norm{\ind{A}\ast \mu_{B'}}_\infty\geq (1+\epsilon/2)\alpha$, or
\item $\norm{\ind{A}\ast \mu_{B''}}_\infty \geq (1+\epsilon/2)\alpha$.
\end{enumerate}
\end{lemma}
\begin{proof}
\uses{reg-conv}
To do.
\end{proof}
