\chapter{Finite field model}

\begin{theorem}\label{ap_in_ff}
If $A_1,A_2,S\subseteq \bbf_q^n$ are such that $A_1$ and $A_2$ both have density at least $\alpha$ then there is a subspace $V$ of codimension

\[\mathrm{codim}(V)\leq 2^{27}\lo(\alpha)^2\lo(\epsilon \alpha)^2\epsilon^{-2}\]
such that
\[\abs{\langle \mu_V\ast \mu_{A_1}\ast \mu_{A_2},\ind{S}\rangle-\langle \mu_{A_1}\ast \mu_{A_2},\ind{S}\rangle}\leq \epsilon.\]
\end{theorem}

\begin{proof}
\uses{chang, linfty_ap_boosted}
(In this proof we write $G=\mathbb{F}_q^n$.) Let $k=\lceil \lo(\epsilon\alpha/4)\rceil$. Note that $\lvert A_1+G\rvert=\lvert G\rvert\leq \alpha^{-1}\lvert A\rvert$. Furthermore, $\lvert A_2\rvert/\lvert S\rvert\geq \alpha$. Therefore by Theorem~\ref{linfty_ap_boosted} there exists some $T\subseteq G$ with
\[\lvert T\rvert \geq \exp(-2^{16}\lo(\alpha)^2k^2\epsilon^{-2})\lvert S\rvert\]
such that
\[\| \mu_T^{(k)}\ast \mu_{A_1}\ast \mu_{A_2}\circ \ind{S}-\mu_{A_1}\ast \mu_{A_2}\circ \ind{S}\|_{\infty}\leq \epsilon/4.\]
Let $\Delta=\Delta_{1/2}(\mu_T)$ and
\[V = \{ x \in G : \gamma(x)=1\textrm{ for all }\gamma\in\Delta\}.\]
Note that
\[\langle \mu_V\ast \mu_{A_1}\ast \mu_{A_2},\ind{S}\rangle = \langle \mu_V,\mu_{A_1}\ast \mu_{A_2}\circ \ind{S}\rangle=\frac{1}{\abs{V}}\sum_{v\in V}\mu_{A_1}\ast \mu_{A_2}\circ \ind{S}(v)\]
and
\[\langle \mu_{A_1}\ast \mu_{A_2},\ind{S}\rangle = \mu_{A_1}\ast \mu_{A_2}\circ\ind{S}(0).\]
Therefore
\[\abs{\langle \mu_V\ast \mu_{A_1}\ast \mu_{A_2},\ind{S}\rangle-\langle \mu_{A_1}\ast \mu_{A_2},\ind{S}\rangle}\leq \frac{1}{\abs{V}}\sum_{v\in V}\abs{\mu_{A_1}\ast \mu_{A_2}\circ \ind{S}(v)-\mu_{A_1}\ast \mu_{A_2}\circ \ind{S}(0)}.\]
In particular it suffices to show that, for any $v\in V$,
\[\abs{\mu_{A_1}\ast \mu_{A_2}\circ \ind{S}(v)-\mu_{A_1}\ast \mu_{A_2}\circ \ind{S}(0)}\leq \epsilon.\]
By the triangle inequality and construction of $T$, it suffices to show that

\[\abs{\mu_T^{(k)}\ast \mu_{A_1}\ast \mu_{A_2}\circ \ind{S}(v)-\mu_T^{(k)}\ast\mu_{A_1}\ast \mu_{A_2}\circ \ind{S}(0)}\leq \epsilon/2.\]
By the Fourier transform we have, for any $x\in G$,
\[\mu_T^{(k)}\ast \mu_{A_1}\ast \mu_{A_2}\circ \ind{S}(x)=\frac{1}{N}\sum_{\gamma\in \widehat{G}}\widehat{\mu_T}(\gamma)^k\widehat{\mu_{A_1}}(\gamma)\widehat{\mu_{A_2}}(\gamma)\widehat{\ind{-S}}(\gamma)\gamma(x).\]
Therefore the left-hand side of the desired inequality is, by the triangle inequality, at most
\[\frac{1}{N}\sum_{\gamma\in \widehat{G}}\abs{\widehat{\mu_T}(\gamma)}^k\abs{\widehat{\mu_{A_1}}(\gamma)\widehat{\mu_{A_2}}(\gamma)\widehat{\ind{-S}}(\gamma)}\abs{\gamma(v)-1}.\]
By choice of $v\in V$ the summand vanishes when $\gamma\in\Delta$. When $\gamma\not\in \Delta$ the summand is bounded above by
\[2^{1-k}\abs{\widehat{\mu_{A_1}}(\gamma)\widehat{\mu_{A_2}}(\gamma)\widehat{\ind{-S}}(\gamma)}.\]
Therefore the left-hand side of the desired inequality is at most
\[2^{1-k}\frac{1}{N}\sum_{\gamma}\abs{\widehat{\mu_{A_1}}(\gamma)\widehat{\mu_{A_2}}(\gamma)\widehat{\ind{-S}}(\gamma)}\leq 2^{1-k}\abs{S}\frac{1}{N}\sum_{\gamma}\abs{\widehat{\mu_{A_1}}(\gamma)\widehat{\mu_{A_2}}(\gamma)}\]
using the trivial bound $\lvert\widehat{\ind{S}}\rvert\leq \abs{S}$. By the Cauchy-Schwarz inequality the sum on the right is at most
\[\brac{\sum_\gamma \abs{\widehat{\mu_{A_1}}}^2}^{1/2}\brac{\sum_\gamma \abs{\widehat{\mu_{A_2}}}^2}^{1/2}.\]
By Parseval's identity this is at most $\alpha^{-1}$. Therefore the desired inequality follows from
\[2^{1-k}\abs{S}\frac{1}{N}\alpha^{-1}\leq 2^{1-k}\alpha^{-1}\leq \epsilon/2.\]

It remains to check the codimension of $V$. For this, let $\Delta'\subseteq \Delta$ be as provided by Chang's lemma, Lemma~\ref{chang}, so that

\[\Delta\subseteq \left\{ \sum_{\gamma\in\Delta'}c_\gamma \gamma : c_\gamma\in \{-1,0,1\} \right\}.\]
Let

\[W= \{ x \in G : \gamma(x)=1\textrm{ for all }\gamma\in\Delta'\}.\]
It follows that $W\leq V$, so it suffices to bound the codimension of $W$. This we can bound trivially using the bound from Chang's lemma and the fact that $\lo(\delta)=\log(e^2/\delta)\leq 2+\log(1/\delta)\leq 4\log(1/\delta)$, provided $\log(1/\delta)\geq 1$, so
\[\abs{\Delta'}\leq  \lceil 4e\mathcal{L}(\delta)\rceil\leq 2^7\log(1/\delta),\]
where
\[\delta=\abs{T}/N\geq \exp(-2^{16}\lo(\alpha)^2k^2\epsilon^{-2}),\]
so
\[\mathrm{codim}(V)\leq \abs{\Delta'}\leq 2^{23}\lo(\alpha)^2k^2\epsilon^{-2}\leq 2^{25}\lo(\alpha)^2\lo(\epsilon \alpha/4)^2\epsilon^{-2},\]
and now use $\lo(\epsilon\alpha/4)\leq 2\lo(\epsilon\alpha)$, say.
\end{proof}


\begin{lemma}
\label{ast_le_circ}
\lean{Lpnorm_conv_le_Lpnorm_dconv}
\leanok
For any function $f:G\to \mathbb{C}$ and integer $k\geq 1$
\[\| f\ast f\|_{2k}\leq \| f\circ f\|_{2k}.\]
\end{lemma}

\begin{proof}
To finish, similar trick to unbalancing.
\end{proof}


\begin{lemma}\label{ast_expand}
\lean{balance_conv}
\leanok
For any function $f$ with $\sum f(x)=1$
\[f\ast f-1/N = (f-1/N)\ast (f-1/N).\]
\end{lemma}

\begin{proof}
\leanok
Expand everything out.
\end{proof}


\begin{lemma}\label{circ_expand}
\lean{balance_dconv}
\leanok
For any function $f$ with $\sum f(x)=1$
\[f\circ f-1/N = (f-1/N)\circ (f-1/N).\]
\end{lemma}

\begin{proof}
\leanok
Expand everything out.
\end{proof}


\begin{lemma}
\label{global_dichotomy}
\lean{global_dichotomy}
\leanok
Let $\epsilon >0$ and $\mu\equiv 1/N$. If $A,C\subseteq G$, where $C$ has density at least $\gamma$, and
\[\abs{N\langle \mu_A\ast \mu_A,\mu_C\rangle -1}>\epsilon\]
then, if $f=(\mu_A-1/N)$, $\norm{f\circ f}_{p(\mu)} \geq \epsilon /2N$ for $p=2\lceil \lo(\gamma)\rceil$.
\end{lemma}

\begin{proof}
\uses{ast_le_circ, ast_expand, circ_expand}
\leanok
By H\"{o}lder's inequality, for any $p\geq 1$
\[\epsilon < \abs{ N\langle \mu_A \ast \mu_A - 1/N, \mu_C \rangle } \leq \norm{\mu_A\ast \mu_A-1/N}_p\gamma^{-1/p}N^{1-1/p}.\]
In particular if we choose $p=2\lceil\lo(\gamma)\rceil$ then $\gamma^{-1/p}\leq e^{1/2}\leq 2$ and so we deduce that, by Lemma~\ref{ast_expand},
\[\norm{f\ast f}_{p}\geq \tfrac{1}{2}\epsilon N^{1/p-1}.\]
It remains to use Lemmas~\ref{ast_expand} and \ref{circ_expand} and apply Lemma~\ref{ast_le_circ}, and note that we can pass from the $L^p$ norm to the $L^p(\mu)$ norm losing a factor of $N^{1/p}$.
\end{proof}


\begin{proposition}\label{di_in_ff}
Let $\epsilon \in (0,1)$. If $A,C\subseteq \bbf_q^n$, where $C$ has density at least $\gamma$, and
\[\abs{N\langle \mu_A\ast \mu_A,\mu_C\rangle -1}> \epsilon\]
then there is a subspace $V$ of codimension 
\[\leq 2^{171}\epsilon^{-24}\lo(\alpha)^4\lo(\gamma)^4.\]
such that $\norm{\ind{A}\ast \mu_V}_\infty \geq (1+\epsilon/32)\alpha$.
\end{proposition}
\begin{proof}
\uses{global_dichotomy,global_drc,unbalancing,ap_in_ff, circ_expand, sift_cor}
By Lemma~\ref{global_dichotomy}, if $f=\mu_A-1/N$,
\[\norm{f\circ f}_{p(\mu)}\geq \epsilon /2N.\]
By Lemma~\ref{unbalancing} there exists some $p'$ such that
\[p'\leq 128\epsilon^{-1}\log(96/\epsilon)\lo(\gamma)\]
such that
\[\norm{f\circ f+1/N}_{p'(\mu)}\geq (1+\epsilon/4)/N.\]
By Lemma~\ref{circ_expand} $f\circ f+1/N=\mu_A\circ \mu_A$.

Let $q=2\lceil p'+2^8\epsilon^{-2}\log(64/\epsilon)\rceil$. By Corollary~\ref{sift_cor}, there are $A_1,A_2$, both of density $\geq \alpha^{2q}$ such that
\[\langle \mu_{A_1}\circ \mu_{A_2},\ind{S}\rangle \geq 1-\epsilon/32\]
where
\[S=\{x : \mu_A\circ \mu_A(x)\geq (1-\epsilon/16)\| \mu_A\circ \mu_A\|_{q(\mu)}\}.\]
Since
\[\| \mu_A\circ \mu_A\|_{q(\mu)}\geq \| \mu_A\circ \mu_A\|_{p'(\mu)}\geq (1+\epsilon/4)/N\]
we know
\[S\subseteq S'=\{x : \mu_A\circ \mu_A(x)\geq (1+\epsilon/8)/N\}.\]

By Theorem~\ref{ap_in_ff} (applied with $\epsilon$ replaced by $\epsilon/32$) there is a subspace $V$ of codimension
\[\leq 2^{37}\lo(\alpha^{2q})^2\lo(\epsilon\alpha^{2q}/32)^2\epsilon^{-2}\] such that
\[\langle \mu_V\ast \mu_{A_1}\circ \mu_{A_2},\ind{S'}\rangle \geq 1-\tfrac{1}{16}\epsilon.\]
Using $\lo(xy)\leq x^{-1}\lo(y)$ we have
\[\lo(\epsilon \alpha^{2q}/32)\leq 32\epsilon^{-1}\lo(\alpha^{2q}),\]
and we also use $\lo(x^y)\leq y\lo(x)$ to simplify the codimension bound to
\[\leq 2^{51}q^4\lo(\alpha)^4\epsilon^{-4}.\]
We further note that (using $\log x\leq x$ say)
\[q\leq 2^{10}p'\epsilon^{-2}\log(64/\epsilon)\leq 2^{30}\epsilon^{-5}\lo(\gamma).\]
Therefore the desired codimension bound follows. Finally, by definition of $S'$, it follows that
\begin{align*}
(1+\epsilon/32)/N
&\leq ((1+\epsilon/8)/N)(1-\epsilon/16)\\
&\leq \langle \mu_V\ast \mu_{A_1}\circ \mu_{A_2},\mu_A\circ \mu_A\rangle\\
&\leq \norm{\mu_V\ast \mu_A}_\infty \norm{\mu_{A}\ast \mu_{A_2}\circ \mu_{A_1}}_1\\
&= \norm{\mu_V\ast \ind{A}}_\infty\abs{A}^{-1},
\end{align*}
and the proof is complete.
\end{proof}


\begin{lemma}
\label{no3aps_inner_prod}
\lean{add_salem_spencer.L2inner_mu_conv_mu_mu_two_smul_mu}
\leanok
If $A\subseteq G$ has no non-trivial three-term arithmetic progressions and $G$ has odd order then
\[\langle \mu_A\ast \mu_A,\mu_{2\cdot A}\rangle = 1/\abs{A}^2.\]
\end{lemma}

\begin{proof}
\leanok
Expand out using definitions.
\end{proof}


\begin{theorem}
Let $q$ be an odd prime power. If $A\subseteq \mathbb{F}_q^n$ with $\alpha=\abs{A}/q^n$ has no non-trivial three-term arithmetic progressions then
\[n \ll \lo(\alpha)^9.\]
\end{theorem}
\begin{proof}
\uses{no3aps_inner_prod, di_in_ff}
Let $t\geq 0$ be maximal such that there is a sequence of subspaces $\bbf_q^n=V_0\geq \cdots \geq V_t$ and associated $A_i\subseteq V_i$ with $A_0=A$ such that
\begin{enumerate}
\item for $0\leq i\leq t$ there exists $x_i$ such that $A_i\subseteq A-x_i$,
\item with $\alpha_i=\abs{A_i}/\abs{V_i}$ we have
\[\alpha_{i+1}\geq \frac{65}{64}\alpha_i\]
for $0\leq i<t$, and
\item \[\mathrm{codim} (V_{i+1}) \leq \mathrm{codim}(V_i)+O(\lo(\alpha)^8)\]
for $0\leq i<t$. (here the second summand should be replaced with whatever explicit codimension bound we get from the above).
\end{enumerate}

Note this is well-defined since $t=0$ meets the requirements, and this process is finite and $t \ll \lo(\alpha)$, since $\alpha_i\leq 1$ for all $i$. Therefore
\[\mathrm{codim}(V_t) \ll \lo(\alpha)^9.\]

Suppose first that
\[\lvert V_t\rvert\langle \mu_{A_t}\ast \mu_{A_t},\mu_{2\cdot A_t}\rangle<1/2.\]
In this case we now apply Proposition~\ref{di_in_ff} to $A_t\subseteq V_t$ with $\epsilon=1/2$ (note that $N=\lvert V_t\rvert$ and all inner product, $\mu$ etc, are relative to the ambient group $V_t$ now). Therefore there is a subspace $V\leq V_t$ of codimension (relative to $V_t$) of $\ll \lo(\alpha)^8$ such that there exists some $x\in V_t$ with
\[\frac{\lvert (A_t-x)\cap V\rvert}{\lvert V\rvert}=1_{A_t}\ast \mu_V(x)=\| 1_{A_t}\ast \mu_V\|_\infty \geq (1+1/64)\alpha_t,\]
which contradicts the maximality of $t$, letting $V_{t+1}=V$ and $A_{t+1}=(A_t-x)\cap V_t$.

Therefore

\[\lvert V_t\rvert\langle \mu_{A_t}\ast \mu_{A_t},\mu_{2\cdot A_t}\rangle\geq 1/2.\]
By Lemma~\ref{no3aps_inner_prod} the left-hand side is equal to $\lvert V_t\rvert/\lvert A_t\rvert^2$, and therefore
\[\alpha^2 \leq \alpha_t^2 \leq 2/\lvert V_t\rvert.\]
By the codimension bound the right-hand side is at most
\[2q^{O(\lo(\alpha)^9)-n}.\]
If $\alpha^2 \leq 2q^{-n/2}$ we are done, otherwise we deduce that $\lo(\alpha)^9 \gg n$ as required.
\end{proof}

\end{document}
